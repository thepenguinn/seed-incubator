\def\seedBaseTotalWidth{6.00}

\def\seedBaseSmallBendLength{0.75}
\def\seedBaseLargeBendLength{1.75}

%% 0 to 90
%% acute angle making with the corresponding bend and the
%% horizontal line to right.
\def\seedBaseSmallAngle{60}
\def\seedBaseLargeAngle{55}

\def\seedBaseCornerRadius{0.05}
\def\seedBaseCornerRadiusInCm{\seedBaseCornerRadius cm}

\def\seedMarkPad{8pt}

\tikzset{
    seedBaseStyle/.style = {
        seedColorGray,
        opacity = 0.5,
        rounded corners = \seedBaseCornerRadiusInCm,
    },
    seedOuterMarkStyle/.style = {
        seedColorLine,
        very thin,
    },
    seedInnerMarkStyle/.style = {
        bgcol,
        thin,
    },
    seedBaseTikzPicStyle/.style = {
        transform shape,
        scale = 0.5,
        %rounded corners = 0.3cm,
        every path/.style = {thick, > = latex},
    },
}

\subtikzpicturedef{subSeedBase} {
    origin,
    north, south, east, west,
    northeast, northwest, southeast, southwest,
    center,
    leftTopCorner,
    leftMostCorner,
    leftBottomCorner,
    rightBottomCorner,
    rightMostCorner,
    rightTopCorner%
} {
    \draw (#1-start) coordinate (#1-origin);
    %%
    \draw
    %% marking coordinates
    (#1-origin) coordinate (#1-leftMostCorner)
    let \p{#1-tmp1} = (\seedBaseSmallAngle:\seedBaseSmallBendLength),
    \p{#1-tmp2} = (\seedBaseLargeAngle:\seedBaseLargeBendLength)
    in
    (#1-leftMostCorner) ++(\seedBaseTotalWidth, {\y{#1-tmp2} - \y{#1-tmp1}})
    coordinate (#1-rightMostCorner)
    %% left corners
    (#1-leftMostCorner) ++(-\seedBaseSmallAngle:\seedBaseSmallBendLength)
    coordinate (#1-leftBottomCorner)
    (#1-leftMostCorner) ++(\seedBaseLargeAngle:\seedBaseLargeBendLength)
    coordinate (#1-leftTopCorner)
    %% right corners
    (#1-rightMostCorner) ++(180 + \seedBaseLargeAngle:\seedBaseLargeBendLength)
    coordinate (#1-rightBottomCorner)
    (#1-rightMostCorner) ++(180 - \seedBaseSmallAngle:\seedBaseSmallBendLength)
    coordinate (#1-rightTopCorner)
    ;
    %%
    \draw
    \markgeocoordinate {#1}
    {(#1-leftTopCorner)} {(#1-leftBottomCorner)}
    {(#1-leftMostCorner)} {(#1-rightMostCorner)}
    ;
}

\subtikzpictureactivate{subSeedBase}

\newcommand\ovrDrawSeedBase[1] {
    (#1-leftTopCorner) -- (#1-leftMostCorner) -- (#1-leftBottomCorner)
    -- (#1-rightBottomCorner) -- (#1-rightMostCorner) -- (#1-rightTopCorner)
    -- cycle
}
