\documentclass[../../main]{subfiles}

\renewcommand\thesection{\arabic{section}}


\begin{document}

\section{NAND Gate} \label{sec:}

In addition to an \emph{inverter}, we also need another logic gate, which is
a \emph{NAND gate}. This will used in the \emph{multiplexer circuit}\footnote{as
a way \emph{enable} and \emph{disable} the multiplexer output.}, as we will be
seeing in the following sections. We will be designing the \emph{NAND gate} just
as we designed the inverter, using the same \emph{BC547} from section \ref{sec:bufferOrInverter}.

\begin{center}
    {\begin{minipage} [c] {0.55\textwidth}
        Figure \ref{fig:clvlNANDGate}
    \end{minipage}
    \hfill
    \begin{minipage} [c] {0.35\textwidth}
        \centering
        \includegraphics [
            max width = \IGXMaxWidth,
            max height = \IGXMaxHeight,
            \IGXDefaultOptionalArgs,
        ] {tikzpics/endClvlNANDGate.pdf}
        \captionof{figure} {A simple RTL NAND gate using \emph{BC547}.}
        \label{fig:clvlNANDGate}
    \end{minipage}\hfill}
\end{center}
\end{document}
