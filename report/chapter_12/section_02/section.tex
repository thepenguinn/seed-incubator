\documentclass[../../main]{subfiles}

\renewcommand\thesection{\arabic{section}}


\begin{document}

\section{EMCU} \label{sec:}

The software side of \esp is implemented using \texttt{esp-idf}, which is build
on top of \emph{FreeRTOS}. By choosing to use \texttt{esp-idf}, we could benefit
from all the goodness of a Real Time OS.

\subsection{Tasks}

So all the things happening on the software side are separated into tasks. A task is
a simple loop that takes care of only one thing. The \emph{FreeRTOS} scheduler will
take care of running these tasks one after the other.

So right now there's two types of task:

\begin{itemize}
    \item Communication task\footnote{tcp server task.}.
    \item Sensor tasks\footnote{task corresponding to each of the sensors}.
\end{itemize}

\subsubsection{Communication Task}

The communication task waits for the WiFi connection and requests and commands from the
SINC app.

\subsubsection{Sensor Tasks}

These tasks will sample the sensor data in a fixed interval and fills into a buffer. These buffers
are guarded by corresponding \emph{mutex locks}\footnote{these tasks are running concurrently, so we need to
protect the reading and writing data to these buffers.}.

\subsection{Drivers to Multiplexer and Relay Board}

Just like the buffers mentioned above we need to guard these hardware resources. So we
just implemented a simple driver for these two boards. Inorder to use any of these
hardware, each task needs to get the access of the corresponding board. This is done
through the driver interface. Once the task finishes using the hardware, it needs to
give back the access.

\subsection{Algorithm}

Here are the things that will happen after a fresh boot up:

\begin{itemize}
    \item Initializes and starts TCP server task.
    \item TCP task will be in blocked state waiting for the WiFi or request event to happen.
    \item Initializes multiplexer circuit.
    \item Initializes relay board circuit.
    \item Initializes and starts each of the sensor tasks.
    \item Sensor tasks will periodically sample the corresponding sensor and fills the buffer.
    \item If a command request from SINC receives, the TCP task will be awaken.
    \item It will serve the request and wait for the closing of the connection.
\end{itemize}

\end{document}
