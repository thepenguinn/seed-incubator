\documentclass[main]{subfiles}

\begin{document}

\def\author{
    Anjana Roy \\
    Aswatheertha T T \\
    Athira Madhusoodanan \\
    Daniel V Mathew \\
}

\def\supervisor{
    Prof. Razeena Yousuf
}

\def\title{Seed Incubation Plant \\ with \\ Environment Monitoring and Control \\}
\def\subtitle{mini project}
\def\subsubtitle{project 01}

\def\authordesc {
    Department of Electronics and Communication \\
    {\bfseries Rajiv Gandhi Institute of Technology} \\
}

\def\supervisordesc {
    Department of Electronics and Communication \\
    {\bfseries Rajiv Gandhi Institute of Technology} \\
}

\def\abstractcontent {
    This project aims to design and implement
    a \emph{low cost} and \emph{semi-automated} \textbf{Seed Incubation Plant}
    based on recent studies
    conducted on \emph{Greenhouse monitoring and
    control systems} based on \textsc{ESP32}.
    The most crucial part of a plant's life is the
    period between when it was a seed and the stage
    it becomes capable of surviving the outer world.
    This period requires the at most care that the
    plant can get. In most of the \emph{third world countries},
    this \emph{incubation} process is still done \emph{manually} by hand.
    In most cases migrating to an automated system is not
    economically feasible, as it may require importing
    various equipments. An \emph{indigenous} implementation
    using a widely available controller is an alternative
    solution, as this project is trying to provide.
    This implementation aims to realize the \emph{atmosphere
    monitoring and control system} which will
    \emph{monitor} and \emph{control} various aspects
    that influence the optimal growth of the seedling,
    including \emph{humidity}, \emph{temperature},
    \emph{lighting}, \emph{soil moisture} etc.
    Wifi capabilities of ESP32 ensures the
    accumulation of collected data and presenting
    it to the user, visually.
    This ESP32 based system offers better \emph{accuracy}
    and \emph{reliability} compared to
    \emph{conventional greenhouse monitoring systems}.
    It can potentially improve crop \emph{yield} and \emph{quality}
    significantly and thus is an essential step towards
    sustainable agriculture.
}

\ifundef{\author}{
    \def\author{Name of Author}
} {}

\ifundef{\supervisor}{
    \def\supervisor{Name of Supervisor}
} {}

\ifundef{\title}{
    \def\title{TITLE OF THE WORK}
} {}

\ifundef{\subtitle}{
    \def\subtitle{COURSE NAME}
} {}

\ifundef{\subsubtitle}{
    \def\subsubtitle{COURSE WORK TYPE}
} {}

\ifundef{\authordesc}{
    \def\authordesc{
        First line \\
        second line \\
        third line \\
    }
} {}

\ifundef{\supervisordesc}{
    \def\supervisordesc{
        First line \\
        second line \\
        third line \\
    }
} {}

\ifundef{\abstractcontent}{
    \def\abstractcontent{
        Say something about what this paper is about.
    }
} {}

\renewcommand{\maketitle} {
    \begin{center}
        %\vspace{-10pt}
        {\Large \scshape \subtitle}
        %\vspace{5pt}
        %\\
        %{\bfseries \subsubtitle}
        %\vspace{10pt}
    \end{center}

    \begin{center}
        \rule{\textwidth}{.75pt}
        %\vspace{2pt}
    \end{center}

    \begin{center}
        {\huge \scshape \title}
        %\vspace{2pt}
    \end{center}

    \begin{center}
        \rule{\textwidth}{.75pt}
        %\vspace{10pt}
    \end{center}

}

\renewenvironment{abstract} {
    %\vspace{6pt}
    \begin{center}
        {\large \bfseries \abstractname}
    \end{center}
    %\vspace{4pt}
} {
}

\newcommand\makeabstractpage[0] {

    \maketitle

    %\vspace{10pt}

    \begin{abstract}
        \begin{center}
            \begin{minipage}{\textwidth}
                \setlength{\parindent}{10pt}
                {\large

                \abstractcontent

                }
            \end{minipage}
        \end{center}
    \end{abstract}

    \vspace{6pt}

    \nocite{ieeeComp, ieeeGreenAI, interEsp}

    \nopagebreak

    \begingroup \let\clearpage\relax
    \printbibliography[title={\large\vspace{-50pt}\centering \small References \vspace{-30pt}}] %Prints bibliography
    \endgroup

    \vspace{6pt}

    \begin{center}
        \begin{minipage}{\textwidth}
            {\large
            \noindent
            \textit{Student Name \& signature}
            \hfill
            \textit{Guide Name \& signature}

            \vspace{10pt}

            \noindent
            {\begin{minipage}[b] {0.4\textwidth}
                \vfill
                \raggedright
                {\linespread{1.2} \itshape \bfseries \author}
            \end{minipage}
            \hfill
            \begin{minipage}[b] {0.4\textwidth}
                \vfill
                \raggedleft
                {\linespread{1.2} \itshape \bfseries \supervisor}
            \end{minipage}}

            \vspace{10pt}

            %\noindent
            %{\begin{minipage}[b] {0.4\textwidth}
            %    \vfill
            %    \raggedright
            %    {\small \itshape \authordesc}
            %\end{minipage}
            %\hfill
            %\begin{minipage}[b] {0.4\textwidth}
            %    \vfill
            %    \raggedleft
            %    {\small \itshape \supervisordesc}
            %\end{minipage}}

            }
        \end{minipage}
    \end{center}
    %\vspace{20pt}
    %\begin{center}
    %    {\LARGE \bfseries
    %    \today
    %    }
    %\end{center}

}


\ifSubfilesClassLoaded {
    \pagenumbering{gobble}
} {
}

\makeabstractpage

\end{document}
