\documentclass[main]{subfiles}

\begin{document}

\def\author{
    1. Anjana Roy \\
    \vspace{3pt}
    2. Aswatheertha T T \\
    \vspace{3pt}
    3. Athira Madhusoodanan \\
    \vspace{3pt}
    4. Daniel V Mathew \\
}

\def\supervisor{
    Prof. Raseena Yousuf
}

\def\subtitle{mini project}
\def\subsubtitle{project 01}

\def\authordesc {
    Department of Electronics and Communication \\
    {\bfseries Rajiv Gandhi Institute of Technology} \\
}

\def\supervisordesc {
    Department of Electronics and Communication \\
    {\bfseries Rajiv Gandhi Institute of Technology} \\
}

\def\abstractcontent {

    \noindent The \textbf{Seed Incubation Plant} with \textbf{Environmental Monitoring and
    Control} attempts to optimize the \emph{germination} and \emph{growth
    process} of \emph{seeds} by providing a \emph{controlled} and
    \emph{monitored environment}.
    %%
    Manual methods are prone to \emph{human error}, requiring constant
    attention to environmental factors which can \emph{fluctuate} and
    \emph{adversely affect seed growth} and may lead to \emph{suboptimal
    results}.
    %%
    In most cases \emph{migrating} to an automated system is not
    \emph{economically feasible}, as it may require importing various
    equipments.Thus, an \emph{indigenous} implementation using a widely
    available \emph{controller} is an alternative solution.
    %%
    This system aims to realize the \emph{environment monitoring and control
    system} which will \emph{monitor} and \emph{control} various aspects that
    influence the optimal growth of the seedling, including \textbf{humidity},
    \textbf{temperature}, \textbf{lighting} and \textbf{soil moisture} by using ESP32
    \emph{microcontroller}. Wifi capabilities of ESP32 ensures
    the accumulation of collected data and presenting it to the user, visually.
    %%
    The solution not only tries to improve the \emph{efficiency} and the
    \emph{success rate of seed germination} but also tries to reduce the
    \emph{labour costs} and \emph{energy consumption}, hoping to offer a
    sustainable and reliable alternative to traditional incubation methods.

}

\ifundef{\author}{
    \def\author{Name of Author}
} {}

\ifundef{\supervisor}{
    \def\supervisor{Name of Supervisor}
} {}

\ifundef{\title}{
    \def\title{TITLE OF THE WORK}
} {}

\ifundef{\subtitle}{
    \def\subtitle{COURSE NAME}
} {}

\ifundef{\subsubtitle}{
    \def\subsubtitle{COURSE WORK TYPE}
} {}

\ifundef{\authordesc}{
    \def\authordesc{
        First line \\
        second line \\
        third line \\
    }
} {}

\ifundef{\supervisordesc}{
    \def\supervisordesc{
        First line \\
        second line \\
        third line \\
    }
} {}

\ifundef{\abstractcontent}{
    \def\abstractcontent{
        Say something about what this paper is about.
    }
} {}

\renewcommand{\maketitle} {
    \begin{center}
        {\LARGE \bfseries \subtitle}
        \vspace{50pt} \\
        {\bfseries \subsubtitle}
        \vspace{10pt}
    \end{center}

    \begin{center}
        \rule{\textwidth}{.75pt}
        \vspace{2pt}
    \end{center}

    \begin{center}
        {\huge \bfseries \title}
        \vspace{2pt}
    \end{center}

    \begin{center}
        \rule{\textwidth}{.75pt}
        \vspace{30pt}
    \end{center}

    \begin{center}
        \begin{minipage}{.86\textwidth}
            {\large
            \noindent
            \textit{Author}
            \hfill
            \textit{Supervisor}

            \vspace{4pt}

            \noindent
            {\begin{minipage}[m] {0.4\textwidth}
                \vfill
                \raggedright
                {\linespread{1.2} \itshape \bfseries \author}
            \end{minipage}
            \hfill
            \begin{minipage}[m] {0.4\textwidth}
                \vfill
                \raggedleft
                {\linespread{1.2} \itshape \bfseries \supervisor}
            \end{minipage}}

            \vspace{15pt}

            \noindent
            {\begin{minipage}[b] {0.4\textwidth}
                \vfill
                \raggedright
                {\small \itshape \authordesc}
            \end{minipage}
            \hfill
            \begin{minipage}[b] {0.4\textwidth}
                \vfill
                \raggedleft
                {\small \itshape \supervisordesc}
            \end{minipage}}

            }
        \end{minipage}
    \end{center}
}

\renewenvironment{abstract} {
    \vspace{30pt}
    \begin{center}
        {\large \bfseries \abstractname}
    \end{center}
    \vspace{4pt}
} {
}

\newcommand\makeabstractpage[0] {

    \maketitle

    \vspace{20pt}

    \begin{abstract}
        \begin{center}
            \begin{minipage}{0.86\textwidth}
                \setlength{\parindent}{20pt}
                {\large

                \abstractcontent

                }
            \end{minipage}
        \end{center}
    \end{abstract}

    \vspace{80pt}
    \begin{center}
        {\LARGE \bfseries
        \today
        }
    \end{center}
}


\ifSubfilesClassLoaded {
    \pagenumbering{gobble}
} {
}

\makeabstractpage
\addcontentsline{toc}{chapter}{Abstract}

\end{document}
