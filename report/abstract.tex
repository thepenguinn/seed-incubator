\documentclass[main]{subfiles}

\begin{document}

\def\author{
    Anjana Roy \\
    Aswatheertha T T \\
    Athira Madhusoodanan \\
    Daniel V Mathew \\
}

\def\supervisor{
    Prof. Razeena Yousuf
}

\def\title{Seed Incubation Plant \\ with \\ Environment Monitoring and Control \\}
\def\subtitle{mini project}
\def\subsubtitle{project 01}

\def\authordesc {
    Department of Electronics and Communication \\
    {\bfseries Rajiv Gandhi Institute of Technology} \\
}

\def\supervisordesc {
    Department of Electronics and Communication \\
    {\bfseries Rajiv Gandhi Institute of Technology} \\
}

\def\abstractcontent {

    The \textbf{Seed Incubation Plant with Environmental Monitoring and
    Control} is designed to optimize the \emph{germination} and \emph{growth
    process} of \textbf{seeds} by providing a \emph{controlled} and
    \emph{monitored environment}.
    %%
    Manual methods are prone to \emph{human error}, requiring constant
    attention to environmental factors which can \emph{fluctuate} and
    \emph{adversely affect} seed growth and may lead to \emph{suboptimal
    results}.
    %%
    In most cases \emph{migrating} to an automated system is not
    \emph{economically feasible}, as it may require importing various
    equipments.Thus, an \emph{indigenous} implementation using a widely
    available controller is an alternative solution.
    %%
    This system aims to realize the atmosphere monitoring and control system
    which will monitor and control various aspects that influence the optimal
    growth of the seedling, including humidity, temperature, lighting and soil
    moisture by using ESP32 and integrating advanced IoT technologies.Wifi
    capabilities of ESP32 ensures the accumulation of collected data and
    presenting it to the user visually.
    %%
    The solution not only improves the efficiency and success rate of seed
    germination but also reduces labor costs and energy consumption, offering a
    sustainable and reliable alternative to traditional incubation methods.
    %%
    This ESP32 based system offers better accuracy and reliability compared to
    conventional greenhouse moni- toring systems. It can potentially improve
    crop yield and quality significantly and thus is an essential step towards
    sustainable agriculture.

    %This project aims to design and implement
    %a \emph{low cost} and \emph{semi-automated}
    %\textbf{Seed Incubation Plant} based
    %on recent studies on \emph{Greenhouse
    %monitoring and control systems} using \textsc{ESP32}.
    %The most crucial part of a plant's life is the
    %period between when it was a seed and the stage
    %it becomes capable of surviving the outer world.
    %This period requires the at most care that the
    %plant can get. In most of the \emph{third world countries},
    %this \emph{incubation} process is still done \emph{manually}
    %by hand. In most cases migrating to an automated system is not
    %economically feasible, as it may require importing
    %various equipments. An \emph{indigenous} implementation
    %using a widely available controller is an alternative
    %solution, as this project is trying to provide.
    %This implementation aims to realize the \emph{atmosphere
    %monitoring and control system} which will
    %\emph{monitor} and \emph{control} various aspects
    %that influence the optimal growth of the seedling,
    %including \emph{humidity}, \emph{temperature},
    %\emph{lighting}, \emph{soil moisture} etc.
    %Wifi capabilities of ESP32 ensures the
    %accumulation of collected data and presenting
    %it to the user, visually.
    %This ESP32 based system offers better \emph{accuracy}
    %and \emph{reliability} compared to
    %\emph{conventional greenhouse monitoring systems}.
    %It can potentially improve crop \emph{yield} and \emph{quality}
    %significantly and thus is an essential step towards
    %sustainable agriculture.

}

\ifundef{\author}{
    \def\author{Name of Author}
} {}

\ifundef{\supervisor}{
    \def\supervisor{Name of Supervisor}
} {}

\ifundef{\title}{
    \def\title{TITLE OF THE WORK}
} {}

\ifundef{\subtitle}{
    \def\subtitle{COURSE NAME}
} {}

\ifundef{\subsubtitle}{
    \def\subsubtitle{COURSE WORK TYPE}
} {}

\ifundef{\authordesc}{
    \def\authordesc{
        First line \\
        second line \\
        third line \\
    }
} {}

\ifundef{\supervisordesc}{
    \def\supervisordesc{
        First line \\
        second line \\
        third line \\
    }
} {}

\ifundef{\abstractcontent}{
    \def\abstractcontent{
        Say something about what this paper is about.
    }
} {}

\renewcommand{\maketitle} {
    \begin{center}
        {\LARGE \bfseries \subtitle}
        \vspace{50pt} \\
        {\bfseries \subsubtitle}
        \vspace{10pt}
    \end{center}

    \begin{center}
        \rule{\textwidth}{.75pt}
        \vspace{2pt}
    \end{center}

    \begin{center}
        {\huge \bfseries \title}
        \vspace{2pt}
    \end{center}

    \begin{center}
        \rule{\textwidth}{.75pt}
        \vspace{30pt}
    \end{center}

    \begin{center}
        \begin{minipage}{.86\textwidth}
            {\large
            \noindent
            \textit{Author}
            \hfill
            \textit{Supervisor}

            \vspace{4pt}

            \noindent
            {\begin{minipage}[m] {0.4\textwidth}
                \vfill
                \raggedright
                {\linespread{1.2} \itshape \bfseries \author}
            \end{minipage}
            \hfill
            \begin{minipage}[m] {0.4\textwidth}
                \vfill
                \raggedleft
                {\linespread{1.2} \itshape \bfseries \supervisor}
            \end{minipage}}

            \vspace{15pt}

            \noindent
            {\begin{minipage}[b] {0.4\textwidth}
                \vfill
                \raggedright
                {\small \itshape \authordesc}
            \end{minipage}
            \hfill
            \begin{minipage}[b] {0.4\textwidth}
                \vfill
                \raggedleft
                {\small \itshape \supervisordesc}
            \end{minipage}}

            }
        \end{minipage}
    \end{center}
}

\renewenvironment{abstract} {
    \vspace{30pt}
    \begin{center}
        {\large \bfseries \abstractname}
    \end{center}
    \vspace{4pt}
} {
}

\newcommand\makeabstractpage[0] {

    \maketitle

    \vspace{20pt}

    \begin{abstract}
        \begin{center}
            \begin{minipage}{0.86\textwidth}
                \setlength{\parindent}{20pt}
                {\large

                \abstractcontent

                }
            \end{minipage}
        \end{center}
    \end{abstract}

    \vspace{80pt}
    \begin{center}
        {\LARGE \bfseries
        \today
        }
    \end{center}
}


\ifSubfilesClassLoaded {
    \pagenumbering{gobble}
} {
}

\makeabstractpage

\end{document}
