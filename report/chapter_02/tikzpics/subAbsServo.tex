\def\absServoWidth{2.5}
\def\absServoHeight{1.5}

\subtikzpicturedef{subAbsServo} {
    t0, b0,
    l0, r0,
    north, south, east, west,
    northeast, northwest, southeast, southwest,
    center,
    origin%
} {
    \draw (#1-start) coordinate (#1-origin);
    \subAbsCtkDiagramBase {#1-absCtkDiagramBase} {#1-origin} {origin}

    \draw [absCtkModule]
    (#1-origin) coordinate (#1-bottomLeftCorner)
    (#1-bottomLeftCorner) ++(\absServoWidth, \absServoHeight)
    coordinate (#1-topRightCorner)

    \markgeocoordinate {#1}
    {(#1-topRightCorner)} {(#1-bottomLeftCorner)}
    {(#1-bottomLeftCorner)} {(#1-topRightCorner)}

    (#1-north) ++(0, \absCtkPinPad)
    coordinate (#1-t0)
    (#1-south) ++(0, -\absCtkPinPad)
    coordinate (#1-b0)

    (#1-west) ++(-\absCtkPinPad, 0)
    coordinate (#1-l0)
    (#1-east) ++(\absCtkPinPad, 0)
    coordinate (#1-r0)

    \markgeocoordinate {#1}
    {(#1-t0)} {(#1-b0)}
    {(#1-l0)} {(#1-r0)}

    (#1-bottomLeftCorner) rectangle (#1-topRightCorner)

    ;
}

\subtikzpictureactivate{subAbsServo}

%% subpicname ptcname
\newcommand\ovrAbsServoDrawLabelName[2] {
    \draw [absCtkModule]
    (#1-center)
    node [
        inner sep = 0pt,
        anchor = center,
    ] {#2}
    ;
}
