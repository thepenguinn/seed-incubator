\documentclass[../../main]{subfiles}

\input{section_header.tex}

\begin{document}

\section{Temperature Monitoring System} \label{sec:tempMonitoringsystem}

For effectively controlling the indoor temperature of the Incubator, a \emph{paired} temperature
sensor would be essential. One will monitor the \emph{outdoor} temperature, as the other monitors
the \emph{indoor} temperature. From the output of each of these sensors, a \emph{relative} temperature
can be estimated and the \emph{heating and cooling system} can be tweaked accordingly.

\subsection{DHT22}

\begin{center}
    {\begin{minipage} [c] {0.55\textwidth}

        First of all we need a \emph{sensor} to measure the temperature reliably.
        \emph{DHT22}, a sensor which can measure both the \emph{relative humidity}\footnote{EMCU will monitor
        and control \emph{air moisture} too.} and \emph{temperature} would be ideal choice for our scenario.
        Another advantage of \emph{DHT22} is that the communication is done via a single pin.
        Please refer the figure \ref{fig:dht22PinImage} for a quick glance at the sensor.

        General and temperature specification:

        \begin{itemize}
            \item \textbf{Operating voltage:} $3.3\si{V}$ to $6\si{V}$.
            \item \textbf{Temperature range:} $-40^\circ$C to $80^\circ$C, with $\pm 0.5^\circ$C tolerance.
            \item \textbf{Temperature resolution:} $0.1^\circ$C.
            \item \textbf{Sensing period:} $2$ seconds.
            %\item \textbf{Humidity range:} $0-100\%$RH with $2\%$ tolerance.
            %\item \textbf{Humidity resolution:} $0.1\%$RH.
        \end{itemize}

    \end{minipage}
    \hfill
    \begin{minipage} [c] {0.35\textwidth}
        \centering
        \includegraphics [
            max width = \IGXMaxWidth,
            max height = \IGXMaxHeight,
            \IGXDefaultOptionalArgs,
        ] {pics/dht22.png}
        \captionof{figure} {
            Pinout of \emph{DHT22} sensor.
            \label{fig:dht22PinImage}
        }
    \end{minipage}\hfill}
\end{center}

Figure \ref{fig:absThermalSensorSystem} shows the connection between

\begin{figure}
    \centering
    \includegraphics [
        max width = \IGXMaxWidth,
        max height = \IGXMaxHeight,
        \IGXDefaultOptionalArgs,
    ] {tikzpics/endAbsThermalSensorSystem.pdf}
    \captionof{figure} {}
    \label{fig:absThermalSensorSystem}
\end{figure}

The temperature can be in a range from .

\end{document}
