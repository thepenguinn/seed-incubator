\documentclass[../../main]{subfiles}

\renewcommand\thesection{\arabic{section}}


\begin{document}

\section{Thermal Control and Exhaust System} \label{sec:}

The \emph{thermals} of the Incubator can be controlled using a \emph{peltier} module, which is
\emph{semi-conductor} based \emph{heat engine}. The COP\footnote{Coefficient of Performance.} is
much less than a typical \emph{compressor} based \emph{heat engines} and is \emph{variable} depending
on the current passing through it. But the compactness of the \emph{peltier} cooler makes it preferable
to this design. One another disadvantage to this type of cooling is that it requires a proper \emph{exhaust}
system. The \emph{peltier} cooler only decrease the temperature of its \emph{cold} side to relative to its
\emph{hot} side\footnote{The \emph{cold} side will be $30 - 40^\circ$C colder than its \emph{hot} side.}.
The \emph{peltier} module is thin, that means the \emph{hot} and \emph{cold} are closer. So it is
\emph{crucial} to properly pull away the heat from its \emph{cold} side as fast as possible.

\vfill

\begin{figure}
    \centering
    \includegraphics [
        max width = \textwidth,
        max height = \textheight,
        \IGXDefaultOptionalArgs,
    ] {tikzpics/endAbsExhaustSystem.pdf}
    \captionof{figure} {}
    \label{fig:absThermalControlAndExhaustSystem}
\end{figure}

\vfill

Figure \ref{fig:absThermalControlAndExhaustSystem} depicts an abstracted diagram of the entire control
system. As we can see that's quiet a lot of actuators to deal with. Using GPIO pins of the \esp directly won't
be feasible. Hence a \emph{shift register} is used in the intermediate control of the actuators.

\end{document}
