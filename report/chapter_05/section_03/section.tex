\documentclass[../../main]{subfiles}

\input{section_header.tex}

\begin{document}

\section{Soil Moisture and Fertilizer Control System} \label{sec:}

The \emph{irrigation system} consist of two parts. There will be a main channel that
goes through the vicinity of each of the seedling. And this channel will have \emph{channel
valves} right above the seedling. The \emph{water valve} and \emph{fertilizer valve}
can pour into the main channel, there by eliminating the need for separate channels.

\pagebreak

\begin{figure} [ht]
    \centering
    \includegraphics [
        max width = \IGXMaxWidth,
        max height = \IGXMaxHeight,
        \IGXDefaultOptionalArgs,
    ] {tikzpics/endAbsSoilMoistureAndFertilizerControlSystem.pdf}
    \captionof{figure} {}
    \label{fig:}
\end{figure}

As with the \emph{thermal system}, this system also uses \emph{shift registers} to
control these valves, thereby reducing the number of pins required to control them.

\end{document}
